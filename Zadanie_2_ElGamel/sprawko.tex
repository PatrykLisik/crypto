\documentclass{article}
    \usepackage[utf8]{inputenc}
    \usepackage{amssymb}
    \usepackage{polski}
    \usepackage[polish]{babel}
    \usepackage{amsmath}
    \usepackage{amsfonts}
    \usepackage{hyperref}
    \usepackage{lastpage}
    \usepackage{graphicx}
    \usepackage{bookmark}
    
    %foot
    \usepackage{fancyhdr}
    \pagestyle{fancyplain}
    \fancyhf{}
    \renewcommand{\headrulewidth}{0pt}
    \renewcommand{\footrulewidth}{0.4pt}
    \fancyfoot[L]{Patryk Lisik, Adam Sadowski, Barbara Morawska \textit{Cukiernia pod wanilinowym uśmieszkiem}}
    \fancyfoot[R]{\thepage\ / \pageref{LastPage}}
    
    
    \begin{document}
    
    \begin{titlepage}
    
    \begin{minipage}{0.33 \textwidth}
    \begin{flushleft}
    \large
    Nr. albumu: \textsc{210254}\linebreak
    Imię i nazwisko:\\
    \textsc{Patryk Lisik}
    \end{flushleft}
    \end{minipage}
    \hspace{0.2\textwidth}
    \begin{minipage}{0.33 \textwidth}
    \begin{flushleft}
    \large
    Nr. albumu: \textsc{210310}\linebreak
    Imię i nazwisko:\\
    \textsc{Adam Sadowski}
    \end{flushleft}
    \end{minipage}

    \vspace{1.5cm}

    \begin{minipage}{0.33 \textwidth}
        \begin{flushleft}
        \large
        Nr. albumu: \textsc{210279}\linebreak
        Imię i nazwisko:\\
        \textsc{Barbara Morawska}
        \end{flushleft}
        \end{minipage}
    
    \vspace{3cm}
    
    \begin{minipage}{0.9\textwidth}
    \begin{flushleft}
    \Large
    Kierunek: \textsc{Informatyka} \\
    rok akademicki: \textsc{2018/2019} \\
    grupa \textsc{poniedziałek 12:30} \linebreak\linebreak
    \end{flushleft}
    \end{minipage}
    
    \vspace{3cm}
    
    {\center\huge\bfseries Kryptografia \par}
    \vspace{1.5cm}
    {\center\huge\bfseries Szyfr ElGamel'a \par}
    
    \end{titlepage}

    \section{Wstęp}
    Kryptosystem ElGamel'a jest szyfrem asymetrycznym bazującym na problemie logarytmu dyskretnego w ciele liczb całkowitych modulo duża liczba pierwsza.

    \section{Zasada działania}

    \begin{enumerate}
        \item Po stronie odbiorcy:
        \begin{enumerate}
            \item Wybór dużej liczby pierwszej -- p
            \item Wybór $\alpha$ takiego, że $ \alpha \in \{ 2,3 \dots p-1 \} $
            \item Wybór klucza prywatnego $k_{pr}$ takiego, że $ k_{pr}=d \in \{2,3 \dots p-2 \} $
            \item Obliczenie klucza publicznego $k_{pb} = \beta \equiv \alpha ^ {k_{pr}} \pmod p$
            \item Publikujemy $(\beta , \alpha ,p)$ jako klucz publiczny
        \end{enumerate}
        \item  Po stronie nadawcy:
        \begin{enumerate}
            \item Wybór klucza prywatnego $i \in \{2,3 \dots ,p-2\}$
            \item Obliczenie klucza publicznego $k_E \equiv \alpha ^ {i} \pmod p$
            \item Obliczanie klucza maskującego/sesji $k_M \equiv \beta ^i \pmod p$
            \item Szyfrowanie wiadomości $Y \equiv X \cdot k_M \pmod p$, gdzie:
            \begin{itemize}
                \item X - niezaszyfrowana wiadomość
                \item Y - zaszyfrowana wiadomość
            \end{itemize}
            \item Publikujemy $(Y,K_E)$
        \end{enumerate}
        \item Po stronie odbiorcy:
        \begin{enumerate}
            \item Obliczanie klucza maskującego/sesji $ k_M \equiv k_m ^ {d} \pmod p $
            \item Deszyfrowanie wiadomości $X \equiv Y \cdot k_M^{-1} \pmod p$, gdzie:
            \begin{itemize}
                \item X - niezaszyfrowana wiadomość
                \item Y - zaszyfrowana wiadomość
            \end{itemize}
            \item Profit
        \end{enumerate}
    \end{enumerate}
    \section{Generowanie dużych liczb pierwszych}
    \section{Implementacja}
    \end{document}